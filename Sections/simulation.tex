\section{First Light And Reionisation Epoch Simulations (\flares)}
\label{sec:flares}

\flares\ is a suite of hydrodynamical zoom simulations that models the formation and evolution of galaxies in the Epoch of Reionisation \citep{Lovell2021, Vijayan2020}. It comprises 40 spherical resimulations, each of radius 14$h^{-1}$Mpc. The zoom regions are selected from a `parent' dark matter only (DMO) simulation of volume (3.2cGpc)$^{3}$, the same as that used in the \ceagle\ simulations \citep{barnes_cluster-eagle_2017}. The large size of the parent box allows a wide range of overdensities to be sampled. To preserve the ordering of overdensities, resimulation regions are selected based on their overdensity at $z=4.67$, when the most extreme overdensities are only mildly non-linear. The resimulations span an overdensity range of $\delta=-0.497\rightarrow0.970$ (see Table A1 of \cite{Lovell2021}), with an over-representation of the densest environments where massive galaxies are more likely to form \citep{chiang_ancient_2013, lovell_characterising_2018}. When building composite distribution functions, a statistical weighting scheme is applied so as to recreate the distribution of environments in the parent box. We refer the reader to \cite{Lovell2021} for a more detailed description of this weighting scheme. The output of the simulations is stored at integer redshifts from $z=15\rightarrow5$.

\flares\ uses the AGNdT9 variant of the \eagle\ subgrid model \citep{schaye_eagle_2015, crain_eagle_2015}, which results in less frequent, more energetic active galactic nuclei (AGN) feedback events. It produces similar mass functions to the fiducial \eagle\ model while better reproducing the observed properties of hot gas in groups and clusters \citep{barnes_cluster-eagle_2017}. We use the same resolution as the fiducial model, with a dark matter particle mass $m_{\mathrm{dm}} = 9.7 \times 10^6\, \mathrm{M}_{\odot}$, initial gas particle mass $m_{\mathrm{g}} = 1.8 \times 10^6\, \mathrm{M}_{\odot}$, and softening length of $2.66\, \mathrm{ckpc}$. The features of the \eagle\ model relevant to this work are discussed in \sec{eagle}. 

The \eagle\ model was calibrated at $z=0$ and agrees well with a number of low redshift observables that were not part of the calibration \citep[e.g.][]{furlong_evolution_2015, Trayford2015, Lagos2015}. It has also shown good agreement with high redshift observations, as seen in previous \flares\ papers, e.g. the galaxy stellar mass function \citep{Lovell2021}, the observed UV luminosity function at $z \geq 5$ \citep{Vijayan2020, Vijayan2022}, HST constraints on galaxy sizes at $z\geq5$ \citep{Roper22}, and the evolution of galaxy colours with redshift \citep{Wilkins22_color}. We have also studied the star formation and metal enrichment histories of galaxies \citep{Wilkins22_metal} and the earliest galaxy populations at $z > 10$ \citep{Wilkins22_frontier}.

\subsection{Structure Finding}
\label{sec:struct_find}