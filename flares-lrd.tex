% mnras_template.tex 
%
% LaTeX template for creating an MNRAS paper
%
% v3.3 released April 2024
% (version numbers match those of mnras.cls)
%
% Copyright (C) Royal Astronomical Society 2015
% Authors:
% Keith T. Smith (Royal Astronomical Society)

% Change log
%
% v3.3 April 2024
%   Updated \pubyear to print the current year automatically
% v3.2 July 2023
%	Updated guidance on use of amssymb package
% v3.0 May 2015
%    Renamed to match the new package name
%    Version number matches mnras.cls
%    A few minor tweaks to wording
% v1.0 September 2013
%    Beta testing only - never publicly released
%    First version: a simple (ish) template for creating an MNRAS paper

%%%%%%%%%%%%%%%%%%%%%%%%%%%%%%%%%%%%%%%%%%%%%%%%%%
% Basic setup. Most papers should leave these options alone.
\documentclass[fleqn,usenatbib]{mnras}

% MNRAS is set in Times font. If you don't have this installed (most LaTeX
% installations will be fine) or prefer the old Computer Modern fonts, comment
% out the following line
\usepackage{newtxtext,newtxmath}
% Depending on your LaTeX fonts installation, you might get better results with one of these:
%\usepackage{mathptmx}
%\usepackage{txfonts}

% Use vector fonts, so it zooms properly in on-screen viewing software
% Don't change these lines unless you know what you are doing
\usepackage[T1]{fontenc}

% Allow "Thomas van Noord" and "Simon de Laguarde" and alike to be sorted by "N" and "L" etc. in the bibliography.
% Write the name in the bibliography as "\VAN{Noord}{Van}{van} Noord, Thomas"
\DeclareRobustCommand{\VAN}[3]{#2}
\let\VANthebibliography\thebibliography
\def\thebibliography{\DeclareRobustCommand{\VAN}[3]{##3}\VANthebibliography}


%%%%% AUTHORS - PLACE YOUR OWN PACKAGES HERE %%%%%

% Only include extra packages if you really need them. Avoid using amssymb if newtxmath is enabled, as these packages can cause conflicts. newtxmatch covers the same math symbols while producing a consistent Times New Roman font. Common packages are:
\usepackage{graphicx}	% Including figure files
\usepackage{amsmath}	% Advanced maths commands

%%%%%%%%%%%%%%%%%%%%%%%%%%%%%%%%%%%%%%%%%%%%%%%%%%

%%%%% AUTHORS - PLACE YOUR OWN COMMANDS HERE %%%%%

% Abbreviated referencing macros
\newcommand{\app}[1]{Appendix~\ref{sec:#1}}
\newcommand{\eq}[1]{Equation~\ref{eq:#1}}
\newcommand{\fig}[1]{Figure~\ref{fig:#1}}
\renewcommand{\sec}[1]{Section~\ref{sec:#1}}
\newcommand{\tab}[1]{Table~\ref{tab:#1}}
\newcommand{\App}[1]{Appendix~\ref{sec:#1}}
\newcommand{\Eq}[1]{Equation~\ref{eq:#1}}
\newcommand{\Fig}[1]{Figure~\ref{fig:#1}}
\newcommand{\Sec}[1]{Section~\ref{sec:#1}}
\newcommand{\Tab}[1]{Table~\ref{tab:#1}}
\newcommand{\ndd}{in prep.}
% Names of things
\newcommand{\bluetides}{{\sc BlueTides}}
\newcommand{\webb}{{\em Webb}}
\newcommand{\HST}{{\em HST}}
\newcommand{\hubble}{{\em Hubble}}
\newcommand{\spitzer}{{\em Spitzer}}

\newcommand{\ceagle}{\mbox{\sc{C-Eagle}}}
\newcommand{\eagle}{\mbox{\sc{Eagle}}}
\newcommand{\lgals}{\mbox{\sc{L-Galaxies}}}
\newcommand{\euclid}{\mbox{\it Euclid}}
\newcommand{\flares}{\mbox{\sc Flares}}
\newcommand{\flare}{\mbox{\sc Flare}}

% Other abbreviations
\newcommand{\SFR}{\mathrm{SFR}}
\newcommand{\zoom}{\mbox{zoom}}
\newcommand{\cloudy}{{\sc cloudy}}

% Units
\newcommand{\Lsun}{\mbox{L$_\odot$}}
\newcommand{\Msun}{\mbox{M$_\odot$}}
\newcommand{\cMpch}{h$^{-1}$ cMpc}

\newcommand{\todo}[1]{{\color{green}***#1***}}
\newcommand{\will}[1]{{\color{purple}***#1***}}
\newcommand{\chris}[1]{{\color{orange}***#1***}}
\newcommand{\peter}[1]{{\color{green}***#1***}}
\newcommand{\steve}[1]{{\color{magenta}***#1***}}
\newcommand{\aswin}[1]{{\color{teal}***#1***}}
\newcommand{\changed}[2]{{\color{red}\sout{#1}}{\color{blue}#2}}
\newcommand{\ammended}[1]{{#1}}

% Please keep new commands to a minimum, and use \newcommand not \def to avoid
% overwriting existing commands. Example:
%\newcommand{\pcm}{\,cm$^{-2}$}	% per cm-squared

%%%%%%%%%%%%%%%%%%%%%%%%%%%%%%%%%%%%%%%%%%%%%%%%%%

%%%%%%%%%%%%%%%%%%% TITLE PAGE %%%%%%%%%%%%%%%%%%%

% Title of the paper, and the short title which is used in the headers.
% Keep the title short and informative.
\title[FLARES N]{First Light And Reionisation Epoch (FLARES) N: An abundance of simulated Little Red Dots}

% The list of authors, and the short list which is used in the headers.
% If you need two or more lines of authors, add an extra line using \newauthor
\author[William J. Roper et al.]{William J. Roper$^{1}$\thanks{E-mail: w.roper@sussex.ac.uk}, %0000-0002-3257-8806
Christopher C. Lovell$^{2}$, % OrCiD: 0000-0001-7964-5933 Christopher c.lovell@herts.ac.uk
Aswin P. Vijayan$^{1}$, % Orcid: 0000-0002-1905-4194 apavi@space.dtu.dk
Louise T. C. Seeyave$^{1}$, %0000-0002-7020-3079 %L.Seeyave@sussex.ac.uk
\newauthor
Peter A. Thomas$^{1}$, % 0000-0001-6888-6483 p.a.thomas@sussex.ac.uk
Stephen M. Wilkins$^{1,6}$ %0000-0003-3903-6935 s.wilkins@sussex.ac.uk
\\
% List of institutions
$^{1}$Astronomy Centre, University of Sussex, Falmer, Brighton BN1 9QH, UK\\
$^{2}$Institute of Cosmology and Gravitation, University of Portsmouth, Burnaby Road, Portsmouth PO1 3FX, UK}

% These dates will be filled out by the publisher
\date{Accepted XXX. Received YYY; in original form ZZZ}

% Prints the current year, for the copyright statements etc. To achieve a fixed year, replace the expression with a number. 
\pubyear{\the\year{}}

% Don't change these lines
\begin{document}
\label{firstpage}
\pagerange{\pageref{firstpage}--\pageref{lastpage}}
\maketitle

% Abstract of the paper
\begin{abstract}
Look at all our Little red dots.
\end{abstract}

% Select between one and six entries from the list of approved keywords.
% Don't make up new ones.
\begin{keywords}
keyword1 -- keyword2 -- keyword3
\end{keywords}

%%%%%%%%%%%%%%%%%%%%%%%%%%%%%%%%%%%%%%%%%%%%%%%%%%

%%%%%%%%%%%%%%%%% BODY OF PAPER %%%%%%%%%%%%%%%%%%

\input{Sections/intro}

\section{First Light And Reionisation Epoch Simulations (\flares)}
\label{sec:flares}

\flares\ is a suite of hydrodynamical zoom simulations that models the formation and evolution of galaxies in the Epoch of Reionisation \citep{Lovell2021, Vijayan2020}. It comprises 40 spherical resimulations, each of radius 14$h^{-1}$Mpc. The zoom regions are selected from a `parent' dark matter only (DMO) simulation of volume (3.2cGpc)$^{3}$, the same as that used in the \ceagle\ simulations \citep{barnes_cluster-eagle_2017}. The large size of the parent box allows a wide range of overdensities to be sampled. To preserve the ordering of overdensities, resimulation regions are selected based on their overdensity at $z=4.67$, when the most extreme overdensities are only mildly non-linear. The resimulations span an overdensity range of $\delta=-0.497\rightarrow0.970$ (see Table A1 of \cite{Lovell2021}), with an over-representation of the densest environments where massive galaxies are more likely to form \citep{chiang_ancient_2013, lovell_characterising_2018}. When building composite distribution functions, a statistical weighting scheme is applied so as to recreate the distribution of environments in the parent box. We refer the reader to \cite{Lovell2021} for a more detailed description of this weighting scheme. The output of the simulations is stored at integer redshifts from $z=15\rightarrow5$.

\flares\ uses the AGNdT9 variant of the \eagle\ subgrid model \citep{schaye_eagle_2015, crain_eagle_2015}, which results in less frequent, more energetic active galactic nuclei (AGN) feedback events. It produces similar mass functions to the fiducial \eagle\ model while better reproducing the observed properties of hot gas in groups and clusters \citep{barnes_cluster-eagle_2017}. We use the same resolution as the fiducial model, with a dark matter particle mass $m_{\mathrm{dm}} = 9.7 \times 10^6\, \mathrm{M}_{\odot}$, initial gas particle mass $m_{\mathrm{g}} = 1.8 \times 10^6\, \mathrm{M}_{\odot}$, and softening length of $2.66\, \mathrm{ckpc}$. The features of the \eagle\ model relevant to this work are discussed in \sec{eagle}. 

The \eagle\ model was calibrated at $z=0$ and agrees well with a number of low redshift observables that were not part of the calibration \citep[e.g.][]{furlong_evolution_2015, Trayford2015, Lagos2015}. It has also shown good agreement with high redshift observations, as seen in previous \flares\ papers, e.g. the galaxy stellar mass function \citep{Lovell2021}, the observed UV luminosity function at $z \geq 5$ \citep{Vijayan2020, Vijayan2022}, HST constraints on galaxy sizes at $z\geq5$ \citep{Roper22}, and the evolution of galaxy colours with redshift \citep{Wilkins22_color}. We have also studied the star formation and metal enrichment histories of galaxies \citep{Wilkins22_metal} and the earliest galaxy populations at $z > 10$ \citep{Wilkins22_frontier}.

\subsection{Structure Finding}
\label{sec:struct_find}

\section{Pure Stellar LRDs}
\label{sec:pure_stellar}

\subsection{Synthesizing stellar spectra}
\label{sec:stellar_fm}

\section{Combined Stellar and AGN LRDs}
\label{sec:combined}

\subsection{Synthesizing AGN spectra}
\label{sec:stellar_fm}

\input{Sections/conclusions}

%%%%%%%%%%%%%%%%%%%%%%%%%%%%%%%%%%%%%%%%%%%%%%%%%%
\section*{Data Availability}

We have data and repos...

\section*{Acknowledgements}

We thank the \eagle\ team for their efforts in developing the \eagle\ simulation code.
We also wish to acknowledge the following open source software packages used in the analysis: \textsf{scipy} \citep{2020SciPy-NMeth}, \textsf{Astropy} \citep{robitaille_astropy:_2013}, and \textsf{matplotlib} \citep{Hunter:2007}.

This work used the DiRAC@Durham facility managed by the Institute for Computational Cosmology on behalf of the STFC DiRAC HPC Facility (www.dirac.ac.uk).
The equipment was funded by BEIS capital funding via STFC capital grants ST/K00042X/1, ST/P002293/1, ST/R002371/1 and ST/S002502/1, Durham University and STFC operations grant ST/R000832/1.
DiRAC is part of the National e-Infrastructure. The \eagle\ simulations were performed using the DiRAC-2 facility at Durham, managed by the ICC, and the PRACE facility Curie based in France at TGCC, CEA, Bruyeres-le-Chatel.

WJR acknowledges support from an STFC consolidated grant... 
CCL acknowledges support from a Dennis Sciama fellowship funded by the University of Portsmouth for the Institute of Cosmology and Gravitation.

We list here the roles and contributions of the authors according to the Contributor Roles Taxonomy (CRediT)\footnote{\url{https://credit.niso.org/}}.
\textbf{William J. Roper}: Conceptualization, Data curation, Methodology, Investigation, Formal Analysis, Visualization, Writing - original draft.
\textbf{Christopher C. Lovell, Aswin P. Vijayan}: Data curation, Writing - review \& editing.
\textbf{Louise Seeyave}: Writing - review \& editing.
Writing - review \& editing.
\textbf{Stephen M. Wilkins}: Conceptualization, Writing - review \& editing.

%%%%%%%%%%%%%%%%%%%% REFERENCES %%%%%%%%%%%%%%%%%%

% The best way to enter references is to use BibTeX:

\bibliographystyle{mnras}
\bibliography{flares} % if your bibtex file is called example.bib

%%%%%%%%%%%%%%%%%%%%%%%%%%%%%%%%%%%%%%%%%%%%%%%%%%

%%%%%%%%%%%%%%%%% APPENDICES %%%%%%%%%%%%%%%%%%%%%

%\appendix


%%%%%%%%%%%%%%%%%%%%%%%%%%%%%%%%%%%%%%%%%%%%%%%%%%


% Don't change these lines
\bsp	% typesetting comment
\label{lastpage}
\end{document}

% End of mnras_template.tex
